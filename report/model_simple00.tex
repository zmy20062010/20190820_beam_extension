% !TEX TS-program = xelatex
% !TEX encoding = UTF-8 Unicode


\documentclass{elsarticle}
\usepackage{amsmath,amssymb,amsfonts}
\usepackage{graphicx}

% \usepackage{geometry}
% \geometry{left = 1.5cm, right = 1.5 cm}

\begin{document}
\title{Investigation upon the performance of piezoelectric energy harvester with flexible extensions}
\author{Maoying Zhou, Weiting Liu}
\maketitle


\begin{abstract}
    hello I begin
\end{abstract}

\section{Model Description}

We seek to investigate the influence of a flexible extension upon the overall performance of a classic piezoelectric cantilever beam energy harvester. In our problem, the energy harvester is comprised of two parts: the primary beam part and the beam extension part, as shown in Figure~\ref{fig:fig_beam_configuration}.
\begin{figure}[!htbp]
    \centering
    \includegraphics[width=0.8\textwidth]{./fig_beam_configuration.pdf}
    \caption{Schematic configuration of the piezoelectric energy harvester with flexible extension.}
    \label{fig:fig_beam_configuration}
\end{figure}

Following the classical analyzing process of piezoelectric bimorph cantilever beams \cite{erturk2009experimentally,park2003dynamics}, we can simply list the following dimensional equations for the piezoelectric primary beam part as:
\begin{equation}
    \left\{\begin{aligned}
        M_p(x_1,t) &= B_p \frac{\partial^2 w_1(x_1,t)}{\partial x_1^2} - e_p V_p (t) \\
        q_p(x_1,t) &= e_p \frac{\partial^2 w_1(x_1,t)}{\partial x_1^2} + \epsilon_p V_p (t),
    \end{aligned}\right.
\end{equation}
where where $M_p(x_1,t)$ is the moment at cross section of $x_1$ and $q_p(x_1,t)$ is the corresponding line charge density on the electrode. $w_1(x_1,t)$ is the displacement function of the primary beam part with $0 \leq x_1 \leq l_p$ and $V_p(t)$ is the voltage across the electrodes. The corresponding coefficients $B_p$, $e_p$, and $\epsilon_p$ are defined as
\begin{equation}
    B_p = \frac{2}{3}b\left\{ E_s h_s^3 + c_{11}^E \left[ (h_s+h_p)^3 - h_s^3 \right] \right\}, \quad e_p = b e_{31}\left(h_s+\frac{1}{2}h_p\right), \quad \epsilon_p = \frac{b \epsilon_{33}^S}{2 h_p}
\end{equation}
in which $c_{11}^E$ and $E_s$ are the elastic constants of the piezoelectric layer and the structure layer, respectively, $e_{31}$ is the piezoelectric charge constant of the piezoelectic layer, $\epsilon_{33}^S$ is the dielectric constant of the piezoelectric layer, $h_s$ and $h_p$ are the half structure layer thickness and piezoelectric layer thickness, respectively, $l_p$ is the length of the primary beam part, and $b$ is the width of the primary beam part.

In terms of the mechanical balance, the equation of a piezoelectric beam can be established using the Euler-Bernoulli assumptions as follows
\begin{equation}
    B_p \frac{\partial^4 w_1(x_1,t)}{\partial x_1^4} + m_p \frac{\partial^2 w_1(x_1,t)}{\partial t^2} = 0
\end{equation}
where $m_p = 2 b(\rho_s h_s + \rho_p h_p)$ is the line mass density of the primary beam part with $\rho_s$ and $\rho_p$ being the volumetric density of the structure layer and the piezoelectric layer, respectively. In turn, principally the piezoelectric energy harvester can be regarded as a current source. So we need to know the charge accumulated on the electrode $Q_p(t)$, which is calculated as 
\begin{equation}
    Q_p(t) = \int_0^{l_p} q_p(t)\ d x_1\ = \ e_p \left.\left[ \frac{\partial w_1(x_1,t)}{\partial x_1} \right]\right|^{l_p}_0 + C_p V_p (t)
\end{equation}
where $C_p = \epsilon_p l_p$ is the inherent capacitance of the piezoelectric layer. According to the Kirchhoff's law, the electric equilibrium equation is 
\begin{equation}
    \frac{d Q_p(t)}{dt}  + \frac{V_p(t)}{R_l} = 0
\end{equation}
where $R_l$ is the externally connected resistive load.

When it comes to the beam extension part ($0 \leq x_2 \leq l_e$), the governing equations are
\begin{equation}
    B_e \frac{\partial^4 w_2(x_2,t)}{\partial x_2^4} + m_e \frac{\partial^2 w_2(x_2,t)}{\partial t^2} = 0
\end{equation}
where $w_2(x_2,t)$ is the displacement of the extension beam at position $0\leq x_2 \leq l_e$, $B_e = \frac{2}{3} b h_e^3$ is the equivalent bending stiffness of the extension beam, $m_e = \rho_e h_e$ is the line mass density of the extension beam, $\rho_e$ is the volumetric mass density of the extension beam, $h_e$ is the half thickness of the extension beam, and $l_e$ is the length of the extension beam. As a result, the defining relations for the cross section moment $M_e(x_2,t)$ at the position $x_2$ is
\begin{equation}
    M_e(x_2,t) = B_e \frac{\partial^2 w_2(x_2,t)}{\partial x_2^2}.
\end{equation}


The related boundary conditions are listed as follows. When $ x_1 = 0$ at the fixed end of the primary beam,
\begin{equation}
    w_1(0,t) = w_b(t), \quad w_1^\prime(0,t) = 0,
\end{equation}
where $w_b(t)$ is the base excitation displacement function. Usually we use a harmonic vibration in the experiment where $w_b(t) = Re \left\{ \xi_b e^{j \sigma t} \right\}$ with $\sigma$ being the angular frequency of the base excitation signal and $j = \sqrt{-1}$ being the imaginary unit. To be more accurate, the amplitude $\xi_b$ is generally set to be a real constant designated by the controller. At the connection point of the primary beam and the beam extension where $x_1 = l_p$ and $x_2 = 0$,
\begin{equation}
    \left\{\begin{aligned}
        w_1(l_p,t) &= w_2(0,t) \\
        \frac{\partial w_1(l_p,t)}{\partial x_1} &= \frac{\partial w_2(0,t)}{\partial x_2} \\
        B_p \frac{\partial^2 w_1(l_p,t)}{\partial x_1^2} - e_p V_p(t) &= B_e \frac{\partial^2 w_2(0,t)}{\partial x_2^2} \\
        B_p \frac{\partial^3 w_1(l_p,t)}{\partial x_1^3} &= B_e \frac{\partial^3 w_2(0,t)}{\partial x_2^3}
    \end{aligned}\right.,
\end{equation}
and at the free end of the beam extension where $x_2 = l_e$, we have
\begin{equation}
    \frac{\partial^2 w_2(l_e,t)}{\partial x_2^2} = 0, \quad \frac{\partial^3 w_2(l_e,t)}{\partial x_2^3} = 0
\end{equation}


\subsection{Harmonic Balance Analysis}
Generally in the literature \cite{erturk2009experimentally,park2003dynamics}, mode decomposition method or finite element method are used to solve the above described equations. Here in this contribution, as we are interested in the steady state response of the piezoelectric energy harvester, and the above described system are linear, harmonic balance method is used. Hence, as a result of the base excitation $w_b(t) = Re \left\{ \xi_b e^{j \sigma t} \right\}$, we can set the steady state response of the displacements $w_1(x_1,t)$ and $w_2(x_2,t)$ of the primary beam and the beam extension respectively as 
\begin{equation}
    w_1(x_1,t) = \tilde{w}_1(x_1)e^{j \sigma t},\quad w_2(x_2,t) = \tilde{w}_2(x_2)e^{j \sigma t},
\end{equation}
the steady state voltage response $V_p(t)$ and charge accumulation $Q_p(t)$ as
\begin{equation}
    V_p(t) = \tilde{V}_p e^{j \sigma t},\quad Q_p(t) = \tilde{Q}_p e^{j \sigma t},
\end{equation}
and the cross section moment $M_p(x_1,t)$ and $M_e(x_2,t)$ described as
\begin{equation}
    M_p(x_1,t) = \tilde{M}_p(x_1) e^{j \sigma t},\quad M_e(x_2,t) = \tilde{M}_e(x_2) e^{j \sigma t}.
\end{equation}
As a result, the system of equations for the piezoelectric energy harvester can be summarized as
\begin{equation}
    \left\{\begin{aligned}
        B_p \frac{\partial^4 \tilde{w}_1(x_1)}{\partial x_1^4} - m_p \sigma^2 \tilde{w}_1(x_1) &= 0 \\
        B_e \frac{\partial^4 \tilde{w}_2(x_2)}{\partial x_2^4} - m_e \sigma^2 \tilde{w}_2(x_2) &= 0 \\
        j \sigma \tilde{Q}_p + \frac{\tilde{V}_p}{R_l} &= 0
    \end{aligned}\right.,
    \label{eq:eq_balance_equations_original}
\end{equation}
\begin{equation}
    \left\{\begin{aligned}
        \tilde{M}_p(x_1) &= B_p \frac{\partial^2 \tilde{w}_1(x_1)}{\partial x_1^2} - e_p \tilde{V}_p \\
        \tilde{Q}_p &= \ e_p \left.\left[ \frac{\partial \tilde{w}_1(x_1)}{\partial x_1} \right]\right|^{l_p}_0 + C_p \tilde{V}_p \\
        \tilde{M}_e(x_2) &= B_e \frac{\partial^2 \tilde{w}_2(x_2)}{\partial x_2^2} 
    \end{aligned}\right.,
    \label{eq:eq_constitutive_equations_original}
\end{equation}
and the boundary conditions become
\begin{equation}
    \left\{\begin{aligned}
        \tilde{w}_1(0) = \xi_b , &\quad \frac{\partial \tilde{w}_1}{\partial x_1} (0) = 0 \\
        w_1(l_p,t) = w_2(0,t), &\quad \frac{\partial \tilde{w}_1(l_p)}{\partial x_1} = \frac{\partial \tilde{w}_2(0)}{\partial x_2} \\
        B_p \frac{\partial^2 \tilde{w}_1(l_p)}{\partial x_1^2} - e_p \tilde{V}_p = B_e \frac{\partial^2 \tilde{w}_2(0)}{\partial x_2^2} , &\quad B_p \frac{\partial^3 \tilde{w}_1(l_p)}{\partial x_1^3} = B_e \frac{\partial^3 \tilde{w}_2(0)}{\partial x_2^3} \\
        \frac{\partial^2 \tilde{w}_2(l_e)}{\partial x_2^2} = 0 , &\quad \frac{\partial^3 \tilde{w}_2(l_e)}{\partial x_2^3} = 0
    \end{aligned}\right..
    \label{eq:eq_boundary_conditions_original}
\end{equation}

From the equations (\ref{eq:eq_balance_equations_original}), (\ref{eq:eq_constitutive_equations_original}), and (\ref{eq:eq_boundary_conditions_original}), we can eliminate the electrical quantities $\tilde{Q}_p$ and $\tilde{V}_p$ by incorporating them into the boundary conditions. Actually, from equations (\ref{eq:eq_balance_equations_original}) and (\ref{eq:eq_constitutive_equations_original}), we have
\begin{equation}
    \tilde{V}_p = \frac{j \sigma R_l e_p}{j \sigma R_l C_p + 1} \left.\left[ \frac{\partial \tilde{w}_1(x_1)}{\partial x_1} \right]\right|^{l_p}_0
\end{equation}
which can actually be used to eliminate the term $\tilde{V}_p$ in the boundary conditions (\ref{eq:eq_boundary_conditions_original}). In the end, we can simplify the problem as a combination of the governing equations
\begin{equation}
    \left\{\begin{aligned}
        B_p \frac{\partial^4 \tilde{w}_1(x_1)}{\partial x_1^4} - m_p \sigma^2 \tilde{w}_1(x_1) &= 0 \\
        B_e \frac{\partial^4 \tilde{w}_2(x_2)}{\partial x_2^4} - m_e \sigma^2 \tilde{w}_2(x_2) &= 0 
    \end{aligned}\right.
    \label{eq:eq_balance_equations_converted}
\end{equation}
and the boundary conditions
\begin{equation}
    \left\{\begin{aligned}
        \tilde{w}_1(0) = \xi_b , &\quad \frac{\partial \tilde{w}_1}{\partial x_1} (0) = 0 \\
        \tilde{w}_1(l_p) = \tilde{w}_2(0), &\quad \frac{\partial \tilde{w}_1(l_p)}{\partial x_1} = \frac{\partial \tilde{w}_2(0)}{\partial x_2} \\
        B_p \frac{\partial^2 \tilde{w}_1(l_p)}{\partial x_1^2} + \frac{j \sigma R_l e_p^2}{j \sigma R_l C_p + 1}  \frac{\partial \tilde{w}_1(l_p)}{\partial x_1} = B_e \frac{\partial^2 \tilde{w}_2(0)}{\partial x_2^2} , &\quad B_p \frac{\partial^3 \tilde{w}_1(l_p)}{\partial x_1^3} = B_e \frac{\partial^3 \tilde{w}_2(0)}{\partial x_2^3} \\
        \frac{\partial^2 \tilde{w}_2(l_e)}{\partial x_2^2} = 0 , &\quad \frac{\partial^3 \tilde{w}_2(l_e)}{\partial x_2^3} = 0
    \end{aligned}\right..
    \label{eq:eq_boundary_conditions_converted}
\end{equation}
which actually manifests as a boundary value problem.

\section{Dimensionless Problem}
Using the following dimensionless group
\begin{equation}
    \tilde{w}_1, \tilde{w}_2 \sim \xi_b,\quad \tilde{x}_1 \sim l_p,\quad \tilde{x}_2 \sim l_e
\end{equation}
we can nondimensionalize the above formulated boundary value problem with respect to the following variables:
\begin{equation}
    \tilde{w}_1 = \xi_b u_1,\quad \tilde{w}_2 = \xi_b u_2,\quad \tilde{x}_1 = l_p x,\quad \tilde{x}_2 = l_e x.
    \label{eq:eq_non_dim_variables}
\end{equation}
Note that here we use one independent space variable $x$ to nondimensionalize two previously used variables $x_1$ and $x_2$. This comes from the fact that the variables $x_1$ and $x_2$ are not coupled with each other in the sense that the primary beam and the extension beam  do not overlap each other except for their joint point where $x_1 = l_p$ and $x_2 = 0$. Thus the two variables do not occur in the equations simultaneously except for the boundary conditions. As for the boundary conditions, the change of variables does not affect the values of the equations. Therefore, the two parts of the piezoelectric energy harvester beam are in fact independent of each other except for the joining point. In one word, the equation (\ref{eq:eq_non_dim_variables}) does not change the problem in essence.

Hence, the above boundary value problem is further changed into the combination of the governing equations
\begin{equation}
    \left\{\begin{aligned}
        \frac{B_p}{l_p^4} u_1^{\prime\prime\prime\prime} - m_p \sigma^2 u_1 &= 0 \\
        \frac{B_e}{l_e^4} u_2^{\prime\prime\prime\prime} - m_e \sigma^2 u_2 &= 0 
    \end{aligned}\right.
    \label{eq:eq_balance_equations_nondim}
\end{equation}
and the boundary conditions
\begin{equation}
    \left\{\begin{aligned}
        u_1(0) = 1 , &\quad u_1^\prime(0) = 0 \\
        u_1(1) = u_2(0), &\quad \frac{1}{l_p} u_1^\prime(1) = \frac{1}{l_e} u_2^\prime(0) \\
        \frac{B_p}{l_p^2} u_1^{\prime\prime}(1) + \frac{j \sigma R_l e_p^2}{j \sigma R_l C_p + 1} \frac{1}{l_p} u_1^{\prime}(1) = \frac{B_e}{l_e^2} u_2^{\prime\prime}(0) , &\quad \frac{B_p}{l_p^3} u_1^{\prime\prime\prime}(1) = \frac{B_e}{l_e^3} u_2^{\prime\prime\prime}(0) \\
        u_2^{\prime\prime}(1) = 0 , &\quad u_2^{\prime\prime\prime}(1) = 0
    \end{aligned}\right..
    \label{eq:eq_boundary_conditions_nondim}
\end{equation}
in which the prime means the derivative with respect to $x$. The equations can again be organized in a more compact form
\begin{equation}
    \left\{\begin{aligned}
         u_1^{\prime\prime\prime\prime} - \nu^2 u_1 &= 0 \\
         u_2^{\prime\prime\prime\prime} - \nu^2 \lambda_m \lambda_l^4 / \lambda_B u_2 &= 0 
    \end{aligned}\right.
    \label{eq:eq_balance_equations_compact}
\end{equation}
and the boundary conditions
\begin{equation}
    \left\{\begin{aligned}
        u_1(0) = 1 , &\quad u_1^\prime(0) = 0 \\
        u_1(1) = u_2(0), &\quad \lambda_l u_1^\prime(1) = u_2^\prime(0) \\
        u_1^{\prime\prime}(1) + \frac{ j \nu \beta }{ j \nu \beta + 1 } \alpha^2 u_1^{\prime}(1) = \lambda_B/ \lambda_l^2 u_2^{\prime\prime}(0) , &\quad u_1^{\prime\prime\prime}(1) = \lambda_B/ \lambda_l^3 u_2^{\prime\prime\prime}(0) \\
        u_2^{\prime\prime}(1) = 0 , &\quad u_2^{\prime\prime\prime}(1) = 0
    \end{aligned}\right..
    \label{eq:eq_boundary_conditions_compact}
\end{equation}
where 
\begin{equation}
    \nu = \sigma \sqrt{ \frac{ m_p l_p^4 }{ B_p } },\quad \lambda_B = \frac{B_e}{B_p},\quad \lambda_m = \frac{m_e}{m_p},\quad \lambda_l = \frac{l_e}{l_p}
\end{equation}
\begin{equation}
    \beta = R_l C_p \sqrt{\frac{B_p}{m_p l_p^4}}, \quad \alpha = e_p \sqrt{\frac{l_p}{C_p B_p}}
\end{equation}

The system (\ref{eq:eq_balance_equations_compact}) and (\ref{eq:eq_boundary_conditions_compact}) is a two-point boundary value problem. The problem can readily be solved by a Chebyschev collocation method using the MATLAB package  \textit{Chebfun} \cite{driscoll2014chebfun}. 
 




\section*{Reference}

\bibliography{maureenchou.bib}
\bibliographystyle{vancouver}


\end{document}



% The conservation of momentum and inextensibility condition for the piezoelectric composite part lead to
% \begin{equation}
%   \begin{aligned}
%       \mu \frac{\partial^2 \mathbf{X}}{\partial T^2} &= \frac{\partial}{\partial S} \left[ F_T \mathbf{e}_\tau - F_N \mathbf{e}_n \right] \\
%       \frac{\partial M}{\partial S} &= F_N \\
%       \frac{\partial \mathbf{X}}{\partial S} &= \mathbf{e}_\tau
%   \end{aligned}
% \end{equation}
% As a result, we have
% \begin{equation}
%   \begin{aligned}
%       \mu \frac{\partial^2 \mathbf{X}}{\partial T^2} &= \frac{\partial}{\partial S} \left[ F_T \mathbf{e}_\tau - \frac{\partial M}{\partial S} \mathbf{e}_n \right] \\
%       \frac{\partial \mathbf{X}}{\partial S} &= \mathbf{e}_\tau
%   \end{aligned}
% \end{equation}