% !TEX TS-program = xelatex
% !TEX encoding = UTF-8
\documentclass{article}

\usepackage{amsmath,amssymb,amsfonts}
\usepackage{graphicx}
\usepackage{xcolor}
\usepackage{geometry}
\geometry{top = 1.5 cm, bottom = 1.5 cm}

\title{Asymptotic analysis of piezoelectric energy harvester}
\author{Maoying Zhou}
\date{\today}

\begin{document}

\maketitle


\section{Summary of the interested equations}

The dynamic equations for a typical piezoelectric composite cantilever beam is 
\begin{equation}
    B_p \frac{\partial^4 w(x,t)}{\partial x^4} + m_p \frac{\partial^2 w(x,t)}{\partial t^2} = 0,
\end{equation}
where $B_p$ is the equivalent bending stiffness and $m_p$ is the line mass density of the piezoelectric cantilever beam. If the piezoelectric elements attached to the cantilever beam is connected to an external electrical load $R_l$, we have
\begin{equation}
    \frac{d Q_p(t)}{d t} + \frac{V_p(t)}{R_l} = 0.
\end{equation}
For the underlying physics, we have the following constitutive equations
\begin{equation}
    \begin{aligned}
        M_p(x,t) &= B_p \frac{\partial^2 w(x,t)}{\partial x^2} - e_p V_p(t), \\
        q_p(x,t) &= e_p \frac{\partial^2 w(x,t)}{\partial x^2} + \varepsilon_p V_p(t),
    \end{aligned}
\end{equation}
or equivalently,
\begin{equation}
    \left\{\begin{aligned}
        M_p(x,t) &= B_p \frac{\partial^2 w(x,t)}{\partial x^2} - e_p V_p(t), \\
        Q_p(x,t) &= e_p \left.\left[ \frac{\partial w(x,t)}{\partial x} \right]\right|_0^{l_p} + C_p V_p(t).
    \end{aligned}\right.
\end{equation}

One end of the cantilever beam is fixed while the other end is free. So the boundary conditions are
\begin{equation}
    \left\{\begin{aligned}
        w(0,t) &= w_b(t), \\
        \frac{\partial w(0,t)}{\partial x} &= 0,
    \end{aligned}\right.
\end{equation}
and
\begin{equation}
    \left\{\begin{aligned}
        M_p(l_p,t) &= B_p \frac{\partial^2 w(l_p,t)}{\partial x^2} - e_p V_p(t) = 0, \\
        Q_p(l_p,t) &= B_p \frac{\partial^3 w(l_p,t)}{\partial x^3} = 0.
    \end{aligned}\right.
\end{equation}


In the classical energy harvesting applications, the cantilever beam is subject to a periodical base excitation 
\begin{equation}
    w_b(t) = \xi_b e^{j \lambda t}
\end{equation}
where $\xi_b$ is usually a real vibration amplitude.





Here we are interested in the classical model of a piezoelectric cantilever beam energy harvester, whose model is described using the following set of equations:
\begin{equation}
    u^{\prime\prime\prime\prime} - \lambda^2 u = 0,
\end{equation}
and the accompanying boundary conditions:
\begin{equation}
    \left\{\begin{aligned}
        u(0) &= 0 \\
        u^\prime(0) &= 0 \\
        u^{\prime\prime}(1) + \frac{j\lambda \beta \alpha^2}{ j\lambda \beta + 1 } u^\prime(1) &= 0 \\
        u^{\prime\prime\prime}(1) &= 0
    \end{aligned}\right.,
\end{equation}
where $\lambda$ is the eigenvalues for the problem, $u$ denotes the displace function of the cantilever beam, $\beta$ is the dimensionless externally connected resistance, and $\alpha$ is the dimensionless piezoelectric coefficient. They can be expressed as follows
\begin{equation}
    \lambda = \omega \sqrt{ \frac{ m_p l_p^4 }{ B_p } }, \quad \beta = R_l C_p \sqrt{\frac{B_p}{m_p l_p^4}}, \quad \alpha = e_p \sqrt{\frac{l_p}{C_p B_p}},
\end{equation}
where $\omega$ is angular frequency, $m_p$ is line mass density, $l_p$ is the length of the cantilever beam, $B_p$ is the bending stiffness, $C_p$ is the inherent capacitance of the piezoelectric layer, $e_p$ is the charge accumulation number, $R_l$ is the externally connected resistance. In practical applications, dielectric property of piezoelectric materials indicate that the parameter $\beta$ is changed from a very small value, which is close to a short-circuit condition to a very large value, which corresponds to an open-circuit condition. Thus we have that $0 \leq \beta \leq \infty$. 


\section{Asymptotic analysis when $\beta$ is small}

Here we seek to find the behavior of the above system at a small value of connected resistance, i.e., $\beta \to 0$. In this case, we set $\beta$ to be the parameter for asymptotic expansion, and 
\begin{equation}
    \begin{aligned}
        \lambda^{(k)} &= \lambda_0^{(k)} + \beta \lambda_1^{(k)} + \beta^2 \lambda_2^{(k)} + \cdots \\
        u^{(k)} &= u_0^{(k)} + \beta u_1^{(k)} + \beta^2 u_2^{(k)} + \cdots
    \end{aligned}
\end{equation}
where $\lambda^{(k)}$ and $u^{(k)}$ are the $k$th eigenvalue and eigenfunction respectively of the above mentioned system under perturbation. $\lambda_0^{(k)}$ and $u_0^{(k)}$ are the corresponding eigenvalue and eigenfunction of the unperturbed system at $\beta = 0$:
\begin{equation}
    u^{\prime\prime\prime\prime} - \lambda_0^2 u = 0,
\end{equation}
\begin{equation}
    \left\{\begin{aligned}
        u(0) &= 0 \\
        u^\prime(0) &= 0 \\
        u^{\prime\prime}(1) &= 0 \\
        u^{\prime\prime\prime}(1) &= 0
    \end{aligned}\right..
\end{equation}

Obviously, the unperturbed system is a classical eigenvalue problem with the eigenvalues determined by 
\begin{equation}
    1 + \cosh(\sqrt{\lambda_0}) \cos(\sqrt{\lambda_0}) = 0
\end{equation}
whose first several values are
\begin{equation}
    \frac{ \sqrt{\lambda_0^{(1)}} }{\pi} = 0.59686,\quad \frac{ \sqrt{\lambda_0^{(2)}} }{\pi} = 1.49418,\quad \frac{ \sqrt{\lambda_0^{(3)}} }{\pi} = 2.50025,\quad \frac{ \sqrt{\lambda_0^{(4)}} }{\pi} = 3.49999, \quad \cdots
\end{equation}


Take the asymptotic expansions and substitute them into the previously derived system of equations, we have the following asymptotic expansions to different orders of $\beta$:

\noindent
$O(\beta^0)$:
\begin{equation}
    \left\{\begin{aligned}
        u_0^{\prime\prime\prime\prime} - \lambda_0^2 u_0 &=0 \\
        u_0(0) &= 0 \\
        u^{\prime}_0(0) &= 0 \\
        u^{\prime\prime}_0(1) &= 0 \\
        u^{\prime\prime\prime}_0(1) &= 0 
    \end{aligned}\right.
\end{equation}
$O(\beta^1)$:
\begin{equation}
    \left\{\begin{aligned}
        u_1^{\prime\prime\prime\prime} - \left( \lambda_0^2 u_1 + 2 \lambda_0 u_0 \lambda_1 \right) &=0 \\
        u_1(0) &= 0 \\
        u^{\prime}_1(0) &= 0 \\
        u^{\prime\prime}_1(1) + j \alpha^2 \lambda_0 u_0^{\prime}(1) &= 0 \\
        u^{\prime\prime\prime}_1(1) &= 0 
    \end{aligned}\right.
\end{equation}
$O(\beta^2)$:
\begin{equation}
    \left\{\begin{aligned}
        u_2^{\prime\prime\prime\prime} - \left( \lambda_0^2 u_2 + 2 \lambda_0 u_1 \lambda_1 + \lambda_1^2 u_0 + 2 \lambda_0 u_0 \lambda_2 \right) &=0 \\
        u_2(0) &= 0 \\
        u^{\prime}_2(0) &= 0 \\
        u^{\prime\prime}_2(1) + \alpha^2 \lambda_0 u_0^{\prime}(1) + j \alpha^2 \left[ \lambda_0 u_1^{\prime}(1) + \lambda_1 u_0^{\prime}(1) \right] &= 0 \\
        u^{\prime\prime\prime}_2(1) &= 0 
    \end{aligned}\right.
\end{equation}


\section{Asymptotic analysis when $\beta$ is large}

Here we seek to find the behavior of the above system at a large value of connected resistance, i.e., $\beta \to \infty$. In this case, we set $\frac{1}{\beta}$ to be the parameter for asymptotic expansion and 
\begin{equation}
    \begin{aligned}
        \lambda^{(k)} &= \tilde\lambda_0^{(k)} + \left( \frac{1}{\beta} \right) \tilde\lambda_1^{(k)} + \left( \frac{1}{\beta} \right)^2 \tilde\lambda_2^{(k)} + \cdots \\
        u^{(k)} &= \tilde{u}_0^{(k)} + \left( \frac{1}{\beta} \right) \tilde{u}_1^{(k)} + \left( \frac{1}{\beta} \right)^2 \tilde{u}_2^{(k)} + \cdots
    \end{aligned}
\end{equation}
where $\tilde\lambda^{(k)}$ and $\tilde{u}^{(k)}$ are the $k$th eigenvalue and eigenfunction respectively of the above mentioned system under perturbation. $\tilde{\lambda}_0^{(k)}$ and $\tilde{u}_0^{(k)}$ are the corresponding eigenvalue and eigenfunction of the unperturbed system at $\beta = \infty$:

\noindent
$O(\frac{1}{\beta^0})$:
\begin{equation}
    \left\{\begin{aligned}
        \tilde{u}_0^{\prime\prime\prime\prime} - \tilde\lambda_0^2 \tilde{u}_0 &=0 \\
        \tilde{u}_0(0) &= 0 \\
        \tilde{u}^{\prime}_0(0) &= 0 \\
        \tilde{u}^{\prime\prime}_0(1) + \alpha^2 \tilde{u}^{\prime}_0(1) &= 0 \\
        \tilde{u}^{\prime\prime\prime}_0(1) &= 0 
    \end{aligned}\right.
\end{equation}
$O(\frac{1}{\beta^1})$:
\begin{equation}
    \left\{\begin{aligned}
        \tilde{u}_1^{\prime\prime\prime\prime} - \left( \tilde\lambda_0^2 u_1 + 2 \tilde\lambda_0 \tilde{u}_0 \tilde\lambda_1 \right) &=0 \\
        \tilde{u}_1(0) &= 0 \\
        \tilde{u}^{\prime}_1(0) &= 0 \\
        \tilde{u}^{\prime\prime}_1(1) + \alpha^2 \tilde{u}_1^{\prime}(1) + \frac{j \alpha^2}{\tilde\lambda_0} \tilde{u}_0^{\prime}(1) &= 0 \\
        \tilde{u}^{\prime\prime\prime}_1(1) &= 0 
    \end{aligned}\right.
\end{equation}
$O(\frac{1}{\beta^2})$:
\begin{equation}
    \left\{\begin{aligned}
        \tilde{u}_2^{\prime\prime\prime\prime} - \left( \tilde\lambda_0^2 \tilde{u}_2 + 2 \tilde\lambda_0 \tilde{u}_1 \tilde\lambda_1 + \tilde\lambda_1^2 \tilde{u}_0 + 2 \tilde\lambda_0 \tilde{u}_0 \tilde\lambda_2 \right) &=0 \\
        \tilde{u}_2(0) &= 0 \\
        \tilde{u}^{\prime}_2(0) &= 0 \\
        \tilde{u}^{\prime\prime}_2(1) + \left[ \alpha^2 \tilde{u}_2^{\prime}(1) - \frac{\alpha^2}{\tilde\lambda_0^2} \tilde{u}_0^{\prime}(1) \right] + j \left[ \frac{\alpha^2}{\tilde\lambda_0} \tilde{u}_1^{\prime}(1) - \frac{\alpha^2 \tilde\lambda_1}{\tilde\lambda_0^2} \tilde{u}_0^{\prime}(1) \right] &= 0 \\
        \tilde{u}^{\prime\prime\prime}_2(1) &= 0 
    \end{aligned}\right.
\end{equation}






\section{Asymptotic analysis in terms of small $\alpha^2$}
Directly using the eigenvalue analysis method for linear boundary value problem, we arrive at the equation for the eigenvalue $\lambda$:
\begin{equation}
    \sqrt{\lambda} \left[ 1 + \left( \frac{e^{\sqrt{\lambda}} + e^{-\sqrt{\lambda}} }{2} \right) \cos \sqrt{\lambda} \right] + \frac{j \beta \lambda \alpha^2}{1 + j \beta \lambda} \left[ \left( \frac{e^{\sqrt{\lambda}} - e^{-\sqrt{\lambda}} }{2} \right) \cos \sqrt{\lambda} + \left( \frac{e^{\sqrt{\lambda}} + e^{-\sqrt{\lambda}} }{2} \right) \sin \sqrt{\lambda} \right] = 0
\end{equation}
or
\begin{equation}
    \sqrt{\lambda} \left[ 1 + \cosh \sqrt{\lambda} \cos \sqrt{\lambda} \right] + \frac{j \beta \lambda \alpha^2}{1 + j \beta \lambda} \left[ \sinh \sqrt{\lambda} \cos \sqrt{\lambda} + \cosh \sqrt{\lambda} \sin \sqrt{\lambda} \right] = 0
\end{equation}

Taking the parameter $\alpha^2$ as the small parameter $\epsilon$ and expanding the eigenvalue $\lambda$ in terms of this $\epsilon$, we have 
\begin{equation}
    \lambda = \lambda_0 + \epsilon \lambda_1 + \epsilon^2 \lambda_2 + \cdots
\end{equation}
and therefore:

\noindent
$O(\epsilon^0)$:
\begin{equation}
    1 + \cosh{\sqrt{\lambda_0}} \cos{\sqrt{\lambda_0}} = 0
\end{equation}
$O(\epsilon^1)$:
\begin{equation}
    2j \beta \lambda_0 \left( \cosh{\sqrt{\lambda_0}} \sin{\sqrt{\lambda_0}} + \sinh{\sqrt{\lambda_0}} \cos{\sqrt{\lambda_0}} \right) + (1+ j \beta \lambda_0)\lambda_1 \left( -\cosh{\sqrt{\lambda_0}} \sin{\sqrt{\lambda_0}} + \sinh{\sqrt{\lambda_0}} \cos{\sqrt{\lambda_0}} \right) = 0
\end{equation}
or equivalently
\begin{equation}
    \lambda_1 = \frac{ 2j \beta \lambda_0 }{ (1+ j \beta \lambda_0) } \frac{ \left( \cosh{\sqrt{\lambda_0}} \sin{\sqrt{\lambda_0}} + \sinh{\sqrt{\lambda_0}} \cos{\sqrt{\lambda_0}} \right) }{ \left( \cosh{\sqrt{\lambda_0}} \sin{\sqrt{\lambda_0}} - \sinh{\sqrt{\lambda_0}} \cos{\sqrt{\lambda_0}} \right) }
\end{equation}






\newpage
\section{Asymptotic analysis in terms of small $\alpha^2$}
The forced vibration problem of a piezoelectric cantilever bimorph is described by
\begin{equation}
    u^{\prime\prime\prime\prime} - \lambda^2 u = \lambda^2,
\end{equation}
and the accompanying boundary conditions:
\begin{equation}
    \left\{\begin{aligned}
        u(0) &= 0, \\
        u^\prime(0) &= 0, \\
        u^{\prime\prime}(1) + \frac{j\lambda \beta }{ j\lambda \beta + 1 } \epsilon u^\prime(1) &= 0, \\
        u^{\prime\prime\prime}(1) &= 0.
    \end{aligned}\right.
\end{equation}

This problem can readily be solved using a conventional boundary value problem solver. Howevere, here we would like to develop an asymptotic expansion of the solution for the system. Using $\epsilon$ as a parameter, we have
\begin{equation}
    u(x;\epsilon) = A_\epsilon \cos{\sqrt{\lambda}x} + B_\epsilon \sin{\sqrt{\lambda}x} + C_\epsilon \cosh{\sqrt{\lambda}x} + D_\epsilon \sinh{\sqrt{\lambda}x} - 1
\end{equation}
As a result, we have
\begin{equation}
    \begin{aligned}
        u^{\prime}(x;\epsilon) &= \sqrt{\lambda} \left( - A_\epsilon \sin{\sqrt{\lambda}x} + B_\epsilon \cos{\sqrt{\lambda}x} + C_\epsilon \sinh{\sqrt{\lambda}x} + D_\epsilon \cosh{\sqrt{\lambda}x} \right) \\
        u^{\prime\prime}(x;\epsilon) &= \lambda \left( - A_\epsilon \cos{\sqrt{\lambda}x} - B_\epsilon \sin{\sqrt{\lambda}x} + C_\epsilon \cosh{\sqrt{\lambda}x} + D_\epsilon \sinh{\sqrt{\lambda}x} \right) \\
        u^{\prime\prime\prime}(x;\epsilon) &= \lambda \sqrt{\lambda} \left( A_\epsilon \sin{\sqrt{\lambda}x} - B_\epsilon \cos{\sqrt{\lambda}x} + C_\epsilon \sinh{\sqrt{\lambda}x} + D_\epsilon \cosh{\sqrt{\lambda}x} \right) \\
    \end{aligned}
\end{equation}

Thus the above boundary value problem is converted into the following linear equation systems:
\begin{equation}
    \left\{\begin{aligned}
        A_\epsilon + C_\epsilon &= 1, \\
        B_\epsilon + D_\epsilon &= 0, \\
        \left( - A_\epsilon \cos{\sqrt{\lambda}} - B_\epsilon \sin{\sqrt{\lambda}} + C_\epsilon \cosh{\sqrt{\lambda}} + D_\epsilon \sinh{\sqrt{\lambda}} \right) &+ \\
        \frac{j \beta \sqrt{\lambda}}{ j\lambda \beta + 1 } \epsilon \left( - A_\epsilon \sin{\sqrt{\lambda}} + B_\epsilon \cos{\sqrt{\lambda}x} + C_\epsilon \sinh{\sqrt{\lambda}x} + D_\epsilon \cosh{\sqrt{\lambda}x} \right) &= 0, \\
        A_\epsilon \sin{\sqrt{\lambda}} - B_\epsilon \cos{\sqrt{\lambda}} + C_\epsilon \sinh{\sqrt{\lambda}} + D_\epsilon \cosh{\sqrt{\lambda}} &= 0.
    \end{aligned}\right.
\end{equation}


Analytically, we can directly obtain the solution to this problem as 
\begin{equation}
    \left\{\begin{aligned}
        A_\epsilon &= \frac{ 1 + \cos\sqrt{\lambda } \cosh\sqrt{\lambda } - \sin\sqrt{\lambda } \sinh\sqrt{\lambda} + \epsilon \frac{2 j \beta \sqrt{\lambda}}{ 1+ j \beta \lambda } \left( \cos\sqrt{\lambda } \sinh\sqrt{\lambda } \right)}{2 \left[ 1 + \cos\sqrt{\lambda } \cosh\sqrt{\lambda } + \epsilon\frac{j \beta \sqrt{\lambda}}{ 1+ j \beta \lambda } \left( \cos\sqrt{\lambda } \sinh\sqrt{\lambda } + \sin\sqrt{\lambda } \cosh\sqrt{\lambda } \right) \right]}, \\
        B_\epsilon &= \frac{ \cos\sqrt{\lambda } \sinh\sqrt{\lambda } + \sin\sqrt{\lambda } \cosh\sqrt{\lambda} + \epsilon \frac{2 j \beta \sqrt{\lambda}}{ 1+ j \beta \lambda } \left( \sin\sqrt{\lambda } \sinh\sqrt{\lambda } \right)}{2 \left[ 1 + \cos\sqrt{\lambda } \cosh\sqrt{\lambda } + \epsilon\frac{j \beta \sqrt{\lambda}}{ 1+ j \beta \lambda } \left( \cos\sqrt{\lambda } \sinh\sqrt{\lambda } + \sin\sqrt{\lambda } \cosh\sqrt{\lambda } \right) \right]}, \\
        C_\epsilon &= \frac{ -1 - \cos\sqrt{\lambda } \cosh\sqrt{\lambda } + \sin\sqrt{\lambda } \sinh\sqrt{\lambda} - \epsilon \frac{2 j \beta \sqrt{\lambda}}{ 1+ j \beta \lambda } \left( \cos\sqrt{\lambda } \sinh\sqrt{\lambda } \right)}{2 \left[ 1 + \cos\sqrt{\lambda } \cosh\sqrt{\lambda } + \epsilon\frac{j \beta \sqrt{\lambda}}{ 1+ j \beta \lambda } \left( \cos\sqrt{\lambda } \sinh\sqrt{\lambda } + \sin\sqrt{\lambda } \cosh\sqrt{\lambda } \right) \right]}, \\
        D_\epsilon &= \frac{ -\cos\sqrt{\lambda } \sinh\sqrt{\lambda } - \sin\sqrt{\lambda } \cosh\sqrt{\lambda} - \epsilon \frac{2 j \beta \sqrt{\lambda}}{ 1+ j \beta \lambda } \left( \sin\sqrt{\lambda } \sinh\sqrt{\lambda } \right)}{2 \left[ 1 + \cos\sqrt{\lambda } \cosh\sqrt{\lambda } + \epsilon\frac{j \beta \sqrt{\lambda}}{ 1+ j \beta \lambda } \left( \cos\sqrt{\lambda } \sinh\sqrt{\lambda } + \sin\sqrt{\lambda } \cosh\sqrt{\lambda } \right) \right]}
    \end{aligned}\right.
\end{equation}


The resulting output voltage $V_p$, current $I_p$, and power $P_p$ can be formulated as follows
\begin{equation}
    \left\{\begin{aligned}
        \tilde{V}_p &= \frac{j \lambda \beta}{j \lambda \beta + 1} \frac{\xi_b}{l_p} \frac{e_p}{C_p} u^\prime(1), \\
        \tilde{I}_p &=  \tilde{V}_p / R_l, \\
        \tilde{P}_p &=  \tilde{V}_p^2 / R_l.
    \end{aligned}\right.
\end{equation}



Using the following regular expansion:
\begin{equation}
    \left\{\begin{aligned}
        A_\epsilon &= A_0 + \epsilon A_1 + \epsilon^2 A_2 + \cdots, \\
        B_\epsilon &= B_0 + \epsilon B_1 + \epsilon^2 B_2 + \cdots, \\
        C_\epsilon &= C_0 + \epsilon C_1 + \epsilon^2 C_2 + \cdots, \\
        D_\epsilon &= D_0 + \epsilon D_1 + \epsilon^2 D_2 + \cdots, 
    \end{aligned}\right.
\end{equation}
we obtain the successive expansion problem:

\noindent
$O(\epsilon^0)$:
\begin{equation}
    \left\{\begin{aligned}
        A_0 + C_0 &= 1, \\
        B_0 + D_0 &= 0, \\
        - A_0 \cos{\sqrt{\lambda}} - B_0 \sin{\sqrt{\lambda}} + C_0 \cosh{\sqrt{\lambda}} + D_0 \sinh{\sqrt{\lambda}} &= 0, \\
        A_0 \sin{\sqrt{\lambda}} - B_0 \cos{\sqrt{\lambda}} + C_0 \sinh{\sqrt{\lambda}} + D_0 \cosh{\sqrt{\lambda}} &= 0.
    \end{aligned}\right.
\end{equation}
The solution is
\begin{equation}
    \left\{\begin{aligned}
        A_0 &= \frac{1 + \cos\sqrt{\lambda} \cosh\sqrt{\lambda} - \sin\sqrt{\lambda} \sinh\sqrt{\lambda} }{2 + 2 \cos\sqrt{\lambda} \cosh\sqrt{\lambda} } \\
        B_0 &= \frac{\cosh\sqrt{\lambda} \sin\sqrt{\lambda} + \cos\sqrt{\lambda} \sinh\sqrt{\lambda} }{2 + 2 \cos\sqrt{\lambda} \cosh\sqrt{\lambda} } \\
        C_0 &= \frac{1 + \cos\sqrt{\lambda} \cosh\sqrt{\lambda} + \sin\sqrt{\lambda} \sinh\sqrt{\lambda} }{2 + 2 \cos\sqrt{\lambda} \cosh\sqrt{\lambda} } \\
        D_0 &= -\frac{\cosh\sqrt{\lambda} \sin\sqrt{\lambda} + \cos\sqrt{\lambda} \sinh\sqrt{\lambda} }{2 + 2 \cos\sqrt{\lambda} \cosh\sqrt{\lambda} }
    \end{aligned}\right.
\end{equation}
Hence we have
\begin{equation}
    - A_0 \sin{\sqrt{\lambda}} + B_0 \cos{\sqrt{\lambda}} + C_0 \sinh{\sqrt{\lambda}} + D_0 \cosh{\sqrt{\lambda}} = \frac{\sinh\sqrt{\lambda }-\sin\sqrt{\lambda }}{\cos\sqrt{\lambda } \cosh\sqrt{\lambda }+1}
\end{equation}

\noindent
$O(\epsilon^1)$:
\begin{equation}
    \left\{\begin{aligned}
        A_1 + C_1 &= 0, \\
        B_1 + D_1 &= 0, \\
        \left( - A_1 \cos{\sqrt{\lambda}} - B_1 \sin{\sqrt{\lambda}} + C_1 \cosh{\sqrt{\lambda}} + D_1 \sinh{\sqrt{\lambda}} \right) &+ \\
        \frac{j \beta \sqrt{\lambda}}{ j\lambda \beta + 1 } \left( - A_0 \sin{\sqrt{\lambda}} + B_0 \cos{\sqrt{\lambda}} + C_0 \sinh{\sqrt{\lambda}} + D_0 \cosh{\sqrt{\lambda}} \right) &= 0, \\
        A_1 \sin{\sqrt{\lambda}} - B_1 \cos{\sqrt{\lambda}} + C_1 \sinh{\sqrt{\lambda}} + D_1 \cosh{\sqrt{\lambda}} &= 0.
    \end{aligned}\right.
\end{equation}
The solution is
\begin{equation}
    \left\{\begin{aligned}
        A_1 &= \frac{j \beta  \sqrt{\lambda }}{1+j \beta  \lambda } \left( \frac{\sinh\sqrt{\lambda }-\sin\sqrt{\lambda }}{\cos\sqrt{\lambda } \cosh\sqrt{\lambda }+1} \right) \left(\frac{\cos\sqrt{\lambda }+\cosh\sqrt{\lambda }}{2 \cos\sqrt{\lambda }\cosh\sqrt{\lambda }+2} \right) \\
        B_1 &= \frac{j \beta  \sqrt{\lambda }}{1+j \beta  \lambda } \left( \frac{\sinh\sqrt{\lambda }-\sin\sqrt{\lambda }}{\cos\sqrt{\lambda } \cosh\sqrt{\lambda }+1} \right) \left( \frac{-\sinh\sqrt{\lambda }+\sin\sqrt{\lambda }}{2 \cos\sqrt{\lambda }\cosh\sqrt{\lambda }+2} \right)\\
        C_1 &= \frac{j \beta  \sqrt{\lambda }}{1+j \beta  \lambda } \left( \frac{\sinh\sqrt{\lambda }-\sin\sqrt{\lambda }}{\cos\sqrt{\lambda } \cosh\sqrt{\lambda }+1} \right) \left( -\frac{\cos\sqrt{\lambda }+\cosh\sqrt{\lambda }}{2 \cos\sqrt{\lambda } \cosh\sqrt{\lambda }+2} \right)\\
        D_1 &= \frac{j \beta  \sqrt{\lambda }}{1+j \beta  \lambda } \left( \frac{\sinh\sqrt{\lambda }-\sin\sqrt{\lambda }}{\cos\sqrt{\lambda } \cosh\sqrt{\lambda }+1} \right) \left( \frac{-\sin\sqrt{\lambda }+\sinh\sqrt{\lambda }}{2 \cos\sqrt{\lambda }\cosh\sqrt{\lambda }+2} \right)
    \end{aligned}\right.
\end{equation}
Then we have
\begin{equation}
    \begin{aligned}
        - A_1 \sin{\sqrt{\lambda}} + B_1 \cos{\sqrt{\lambda}} + C_1 \sinh{\sqrt{\lambda}} + D_1 \cosh{\sqrt{\lambda}} \\
        = \frac{j \beta  \sqrt{\lambda }}{1+j \beta  \lambda } \left(\frac{\sin\sqrt{\lambda } -\sinh\sqrt{\lambda }}{\cos\sqrt{\lambda } \cosh\sqrt{\lambda }+1} \right)  \left( \frac{\cos\sqrt{\lambda } \sinh\sqrt{\lambda }+\sin\sqrt{\lambda } \cosh\sqrt{\lambda }}{\cos\sqrt{\lambda } \cosh\sqrt{\lambda }+1} \right)
    \end{aligned}
\end{equation}

\noindent
$O(\epsilon^2)$:
\begin{equation}
    \left\{\begin{aligned}
        A_2 + C_2 &= 0, \\
        B_2 + D_2 &= 0, \\
        \left( - A_2 \cos{\sqrt{\lambda}} - B_2 \sin{\sqrt{\lambda}} + C_2 \cosh{\sqrt{\lambda}} + D_2 \sinh{\sqrt{\lambda}} \right) &+ \\
        \frac{j \beta \sqrt{\lambda}}{ j\lambda \beta + 1 } \left( - A_1 \sin{\sqrt{\lambda}} + B_1 \cos{\sqrt{\lambda}} + C_1 \sinh{\sqrt{\lambda}} + D_1 \cosh{\sqrt{\lambda}} \right) &= 0, \\
        A_2 \sin{\sqrt{\lambda}} - B_2 \cos{\sqrt{\lambda}} + C_2 \sinh{\sqrt{\lambda}} + D_2 \cosh{\sqrt{\lambda}} &= 0.
    \end{aligned}\right.
\end{equation}
The solution is
\scriptsize
\begin{equation}
    \left\{\begin{aligned}
        A_2 &= \left( \frac{j \beta \sqrt{\lambda }}{1+j \beta \lambda } \right)^2 \left(\frac{ \sinh\sqrt{\lambda } - \sin\sqrt{\lambda }}{\cos\sqrt{\lambda } \cosh\sqrt{\lambda }+1} \right) \left( \frac{\cos\sqrt{\lambda } \sinh\sqrt{\lambda }+\sin\sqrt{\lambda } \cosh\sqrt{\lambda }}{\cos\sqrt{\lambda } \cosh\sqrt{\lambda }+1} \right) \left(\frac{\cos\sqrt{\lambda }+\cosh\sqrt{\lambda }}{2 \cos\sqrt{\lambda }\cosh\sqrt{\lambda }+2} \right) \\
        B_2 &= \left( \frac{j \beta \sqrt{\lambda }}{1+j \beta \lambda } \right)^2 \left(\frac{\sinh\sqrt{\lambda } - \sin\sqrt{\lambda }}{\cos\sqrt{\lambda } \cosh\sqrt{\lambda }+1} \right) \left( \frac{\cos\sqrt{\lambda } \sinh\sqrt{\lambda }+\sin\sqrt{\lambda } \cosh\sqrt{\lambda }}{\cos\sqrt{\lambda } \cosh\sqrt{\lambda }+1} \right) \left( \frac{-\sinh\sqrt{\lambda }+\sin\sqrt{\lambda }}{2 \cos\sqrt{\lambda }\cosh\sqrt{\lambda }+2} \right)\\
        C_2 &= \left( \frac{j \beta \sqrt{\lambda }}{1+j \beta \lambda } \right)^2 \left(\frac{\sinh\sqrt{\lambda } - \sin\sqrt{\lambda }}{\cos\sqrt{\lambda } \cosh\sqrt{\lambda }+1} \right) \left( \frac{\cos\sqrt{\lambda } \sinh\sqrt{\lambda }+\sin\sqrt{\lambda } \cosh\sqrt{\lambda }}{\cos\sqrt{\lambda } \cosh\sqrt{\lambda }+1} \right) \left( -\frac{\cos\sqrt{\lambda }+\cosh\sqrt{\lambda }}{2 \cos\sqrt{\lambda } \cosh\sqrt{\lambda }+2} \right)\\
        D_2 &= \left( \frac{j \beta \sqrt{\lambda }}{1+j \beta \lambda } \right)^2 \left(\frac{\sinh\sqrt{\lambda } - \sin\sqrt{\lambda }}{\cos\sqrt{\lambda } \cosh\sqrt{\lambda }+1} \right) \left( \frac{\cos\sqrt{\lambda } \sinh\sqrt{\lambda }+\sin\sqrt{\lambda } \cosh\sqrt{\lambda }}{\cos\sqrt{\lambda } \cosh\sqrt{\lambda }+1} \right) \left( \frac{-\sin\sqrt{\lambda }+\sinh\sqrt{\lambda }}{2 \cos\sqrt{\lambda }\cosh\sqrt{\lambda }+2} \right)
    \end{aligned}\right.
\end{equation}
\normalsize

To get higher order expansions, we can use the following iteration method:

\noindent
$O(\epsilon^{k+1})\ (k\geq 1)$:
\begin{equation}
    \left\{\begin{aligned}
        A_{k+1} + C_{k+1} &= 0, \\
        B_{k+1} + D_{k+1} &= 0, \\
        \left( - A_{k+1} \cos{\sqrt{\lambda}} - B_{k+1} \sin{\sqrt{\lambda}} + C_{k+1} \cosh{\sqrt{\lambda}} + D_{k+1} \sinh{\sqrt{\lambda}} \right) &+ \\
        \frac{j \beta \sqrt{\lambda}}{ j\lambda \beta + 1 } \left( - A_{k} \sin{\sqrt{\lambda}} + B_{k} \cos{\sqrt{\lambda}} + C_{k} \sinh{\sqrt{\lambda}} + D_{k} \cosh{\sqrt{\lambda}} \right) &= 0, \\
        A_{k+1} \sin{\sqrt{\lambda}} - B_{k+1} \cos{\sqrt{\lambda}} + C_{k+1} \sinh{\sqrt{\lambda}} + D_{k+1} \cosh{\sqrt{\lambda}} &= 0.
    \end{aligned}\right.
\end{equation}
The solution is
\begin{equation}
    \left\{\begin{aligned}
        A_{k+1} &= \left( \frac{j \beta \sqrt{\lambda }}{1+j \beta \lambda } \right) \left(\frac{\cos\sqrt{\lambda }+\cosh\sqrt{\lambda }}{2 \cos\sqrt{\lambda }\cosh\sqrt{\lambda }+2} \right) \left( Q_k \right) \\
        B_{k+1} &= \left( \frac{j \beta \sqrt{\lambda }}{1+j \beta \lambda } \right) \left( \frac{-\sinh\sqrt{\lambda }+\sin\sqrt{\lambda }}{2 \cos\sqrt{\lambda }\cosh\sqrt{\lambda }+2} \right) \left( Q_k \right) \\
        C_{k+1} &= \left( \frac{j \beta \sqrt{\lambda }}{1+j \beta \lambda } \right) \left( -\frac{\cos\sqrt{\lambda }+\cosh\sqrt{\lambda }}{2 \cos\sqrt{\lambda } \cosh\sqrt{\lambda }+2} \right) \left( Q_k \right) \\
        D_{k+1} &= \left( \frac{j \beta \sqrt{\lambda }}{1+j \beta \lambda } \right) \left( \frac{-\sin\sqrt{\lambda }+\sinh\sqrt{\lambda }}{2 \cos\sqrt{\lambda }\cosh\sqrt{\lambda }+2} \right) \left( Q_k \right) 
    \end{aligned}\right.
\end{equation}
where for $k \geq 2$
\begin{equation}
    Q_k = - A_{k} \sin{\sqrt{\lambda}} + B_{k} \cos{\sqrt{\lambda}} + C_{k} \sinh{\sqrt{\lambda}} + D_{k} \cosh{\sqrt{\lambda}},
\end{equation}
and for $k \geq 0$
\begin{equation}
    \begin{aligned}
        Q_{k+1} &= - A_{k+1} \sin{\sqrt{\lambda}} + B_{k+1} \cos{\sqrt{\lambda}} + C_{k+1} \sinh{\sqrt{\lambda}} + D_{k+1} \cosh{\sqrt{\lambda}} \\
        &= - \left( \frac{ \sin\sqrt{\lambda} \cosh\sqrt{\lambda} + \cos\sqrt{\lambda} \sinh\sqrt{\lambda} }{ \cos\sqrt{\lambda }\cosh\sqrt{\lambda }+1 } \right) \left( \frac{j \beta \sqrt{\lambda }}{1+j \beta \lambda } \right) Q_k,
    \end{aligned}
\end{equation}
and
\begin{equation}
    \begin{aligned}
        Q_{1} &= - A_1 \sin{\sqrt{\lambda}} + B_1 \cos{\sqrt{\lambda}} + C_1 \sinh{\sqrt{\lambda}} + D_1 \cosh{\sqrt{\lambda}} \\
        &= \frac{j \beta  \sqrt{\lambda }}{1+j \beta  \lambda } \left(\frac{\sin\sqrt{\lambda } -\sinh\sqrt{\lambda }}{\cos\sqrt{\lambda } \cosh\sqrt{\lambda }+1} \right)  \left( \frac{\cos\sqrt{\lambda } \sinh\sqrt{\lambda }+\sin\sqrt{\lambda } \cosh\sqrt{\lambda }}{\cos\sqrt{\lambda } \cosh\sqrt{\lambda }+1} \right)
    \end{aligned}
\end{equation}
\begin{equation}
    Q_0 = \frac{\sinh\sqrt{\lambda }-\sin\sqrt{\lambda }}{\cos\sqrt{\lambda } \cosh\sqrt{\lambda }+1}
\end{equation}

Hence it is shown that for $k \geq 0$
\begin{equation}
    \begin{aligned}
        Q_{k} &= - \left( \frac{ \sin\sqrt{\lambda} \cosh\sqrt{\lambda} + \cos\sqrt{\lambda} \sinh\sqrt{\lambda} }{ \cos\sqrt{\lambda }\cosh\sqrt{\lambda }+1 } \right) \left( \frac{j \beta \sqrt{\lambda }}{1+j \beta \lambda } \right) Q_k \\
        &= \left[- \left( \frac{j \beta \sqrt{\lambda }}{1+j \beta \lambda } \right) \left( \frac{ \sin\sqrt{\lambda} \cosh\sqrt{\lambda} + \cos\sqrt{\lambda} \sinh\sqrt{\lambda} }{ \cos\sqrt{\lambda }\cosh\sqrt{\lambda }+1 } \right)  \right]^k \left( \frac{\sinh\sqrt{\lambda }-\sin\sqrt{\lambda }}{\cos\sqrt{\lambda } \cosh\sqrt{\lambda }+1} \right)
    \end{aligned}
\end{equation}

As a result, we obtain that for $k \geq 0$
\scriptsize
\begin{equation}
    \left\{\begin{aligned}
        A_{k+1} &= \left( \frac{j \beta \sqrt{\lambda }}{1+j \beta \lambda } \right)^{k+1} \left( \frac{ -\sin\sqrt{\lambda} \cosh\sqrt{\lambda} - \cos\sqrt{\lambda} \sinh\sqrt{\lambda} }{ \cos\sqrt{\lambda }\cosh\sqrt{\lambda }+1 } \right)^k \left( \frac{\sinh\sqrt{\lambda }-\sin\sqrt{\lambda }}{\cos\sqrt{\lambda } \cosh\sqrt{\lambda }+1} \right) \left(\frac{\cos\sqrt{\lambda }+\cosh\sqrt{\lambda }}{2 \cos\sqrt{\lambda }\cosh\sqrt{\lambda }+2} \right) \\
        B_{k+1} &= \left( \frac{j \beta \sqrt{\lambda }}{1+j \beta \lambda } \right)^{k+1}  \left( \frac{ -\sin\sqrt{\lambda} \cosh\sqrt{\lambda} - \cos\sqrt{\lambda} \sinh\sqrt{\lambda} }{ \cos\sqrt{\lambda }\cosh\sqrt{\lambda }+1 } \right)^k \left( \frac{\sinh\sqrt{\lambda }-\sin\sqrt{\lambda }}{\cos\sqrt{\lambda } \cosh\sqrt{\lambda }+1} \right) \left( \frac{-\sinh\sqrt{\lambda }+\sin\sqrt{\lambda }}{2 \cos\sqrt{\lambda }\cosh\sqrt{\lambda }+2} \right) \\
        C_{k+1} &= \left( \frac{j \beta \sqrt{\lambda }}{1+j \beta \lambda } \right)^{k+1}  \left( \frac{ -\sin\sqrt{\lambda} \cosh\sqrt{\lambda} - \cos\sqrt{\lambda} \sinh\sqrt{\lambda} }{ \cos\sqrt{\lambda }\cosh\sqrt{\lambda }+1 } \right)^k \left( \frac{\sinh\sqrt{\lambda }-\sin\sqrt{\lambda }}{\cos\sqrt{\lambda } \cosh\sqrt{\lambda }+1} \right) \left( \frac{-\cos\sqrt{\lambda }-\cosh\sqrt{\lambda }}{2 \cos\sqrt{\lambda } \cosh\sqrt{\lambda }+2} \right) \\
        D_{k+1} &= \left( \frac{j \beta \sqrt{\lambda }}{1+j \beta \lambda } \right)^{k+1} \left( \frac{ -\sin\sqrt{\lambda} \cosh\sqrt{\lambda} - \cos\sqrt{\lambda} \sinh\sqrt{\lambda} }{ \cos\sqrt{\lambda }\cosh\sqrt{\lambda }+1 } \right)^k \left( \frac{\sinh\sqrt{\lambda }-\sin\sqrt{\lambda }}{\cos\sqrt{\lambda } \cosh\sqrt{\lambda }+1} \right) \left( \frac{-\sin\sqrt{\lambda }+\sinh\sqrt{\lambda }}{2 \cos\sqrt{\lambda }\cosh\sqrt{\lambda }+2} \right)
    \end{aligned}\right.
\end{equation}
\normalsize

























































\end{document}